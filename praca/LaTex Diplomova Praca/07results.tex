\chapter{Výsledky}\label{chap:conclusion}
Verikikačný nástroj sme použili na dva programy. Prvý je GCD 
(Greatest common divisor) a druhý Bubble sort. Pre oba tieto programy budeme chcieť
verifikovať vlastnosti programu, t. j. FP a leads-to. 

Na overenie vlastností sme použili nástroj NuSMV. Príkazom 
\texttt{bin/NuSMV -int public/out/program.smv} sme sa dostali do interaktívneho shell-u. 
Nasledujúce príkazy v shell-i nás doviedli k výsledkom, z ktorých sme dokázali FP a leads-to 
overiť. Príkaz \texttt{show\_traces -v} nám vrátil všetky vykonané kroky. 
Pre overenie vlastností sme uviedli iba niektoré časti krokov.

\begin{lstlisting}
    NuSMV > go
    NuSMV > pick_state -r
    NuSMV > simulate -r -k 3
    NuSMV > show_traces -v
\end{lstlisting}

\section{GCD}
GCD je jednoduchý algoritmus na výpočet najväčšieho spoločného deliteľa.
\begin{lstlisting}
    program         GCD
        declare     x: integer, 
                    y: integer
        always      decx, decy := x > y, y > x
        initially   x, y : 60, 48
        assign      x := x - y if decx
            []      y := y - x if decy
    end
\end{lstlisting}

Po vykonaní 38 krokoch hodnoty x a y nadobudli rovnakú hodnotu. To znamená, že spoločný
deliteľ bol nájdený.

\begin{lstlisting}
    -> State: 1.22 <-
    x = 0sd32_12
    y = 0sd32_48
    .
    -> State: 1.26 <-
    x = 0sd32_12
    y = 0sd32_36
    ...
    -> State: 1.39 <-
    x = 0sd32_12
    y = 0sd32_12
    ...
    -> State: 1.41 <-
    x = 0sd32_12
    y = 0sd32_12
\end{lstlisting}

Od kroku 1.39 sa hodnoty nezmenili, teda \textbf{nastáva FP}. Pre overenie vlastnosti 
leads-to sme použili nasledujúci výraz.
\begin{ceqn}
    \begin{align*}
        x \geq y \rightarrow y \geq x
    \end{align*}
\end{ceqn}

Výraz platil vo všetkých krokoch teda \textbf{leads-to je splnené}.


\section{Bubble sort}
Bubble sort je algoritmus, ktorý slúži na utriedenie jednorozmerného poľa čísel. Jeho princípom je posúvať
neutriedené čísla od najmenšieho po najväčšie. Využíva výmenný triediaci algoritmus, ak je
jedno číslo väčšie ako to druhé tak im zamení miesto v poli. 

V našom konkrétnom príklade má pole veľkosť N, kde N je 5, a čísla nadobúdali náhodnú hodnotu. 
UNITY program v sekcii \textbf{assign} vytvorí $N-1$ priradení, ktoré kontrolujú či sa dané 
dve čísla môžu zameniť.

\begin{lstlisting}
    program         BubbleSort
        declare     N: integer, 
                    a: array [0..N] of integer
        initially   N: 5 []
                    <|| k: 0 <= k < N :: a[k] = rand() >
        assign      <[] i: 0 <= i < N - 1 :: 
                        a[i], a[i + 1] = a[i + 1], a[i] 
                        if a[i] > a[i + 1] >
    end
\end{lstlisting}

Po spustení programu nám verfikátor vygeneroval pole \texttt{a}. 

\begin{lstlisting}
    a = [62, 72, 9, 83, 59]
\end{lstlisting}

Na začiatok sme simulovali 6 krokov.

\begin{lstlisting}
    // tu môžeme vidieť inicializáciu poľa
    -> State: 1.4 <-
    tmp = 0sd32_0
    N = 0sd32_5
    a[0] = 0sd32_62
    a[1] = 0sd32_72
    a[2] = 0sd32_0
    a[3] = 0sd32_0
    a[4] = 0sd32_0
    .
    // pole bolo inicializované v kroku 1.7, po opakovaní príkazu simulate -r -k 3
    -> State: 1.7 <-
    tmp = 0sd32_0
    N = 0sd32_5
    a[0] = 0sd32_62
    a[1] = 0sd32_72
    a[2] = 0sd32_9
    a[3] = 0sd32_83
    a[4] = 0sd32_59
\end{lstlisting}

Pre ďalšie výsledky sme simulovali program o ďalších 10 krokov.
\begin{lstlisting}
    // premenná tmp, ktorá slúži na usporiadanie poľa nadobúda hodnotu 83, čo znamená, že proces triedenia sa už začal
    -> State: 1.17 <-
    tmp = 0sd32_83
    N = 0sd32_5
    a[0] = 0sd32_62
    a[1] = 0sd32_72
    a[2] = 0sd32_9
    a[3] = 0sd32_59
    a[4] = 0sd32_83
\end{lstlisting}

Pre ďalšie výsledky sme simulovali program o 20 krokov.
\begin{lstlisting}
    // pole sa už začína utriedovať
    -> State: 1.29 <-
    tmp = 0sd32_72
    N = 0sd32_5
    a[0] = 0sd32_62
    a[1] = 0sd32_9
    a[2] = 0sd32_59
    a[3] = 0sd32_72
    a[4] = 0sd32_83
    .
    -> State: 1.37 <-
    tmp = 0sd32_62
    N = 0sd32_5
    a[0] = 0sd32_9
    a[1] = 0sd32_62
    a[2] = 0sd32_59
    a[3] = 0sd32_72
    a[4] = 0sd32_83
\end{lstlisting}

Pole bolo utriedené po vykonaní 68 krokoch, každý ďalší krok sa už pole nezmenilo a 
\textbf{nastáva FP}, kedy sa už žiadne vstupné hodnoty nezmenia.

\begin{lstlisting}
    -> State: 1.69 <-
    tmp = 0sd32_62
    N = 0sd32_5
    a[0] = 0sd32_9
    a[1] = 0sd32_59
    a[2] = 0sd32_62
    a[3] = 0sd32_72
    a[4] = 0sd32_83
    .
    -> State: 1.77 <-
    tmp = 0sd32_62
    N = 0sd32_5
    a[0] = 0sd32_9
    a[1] = 0sd32_59
    a[2] = 0sd32_62
    a[3] = 0sd32_72
    a[4] = 0sd32_83
\end{lstlisting}



Pre overenie vlastnosti leads-to sme použili nasledujúci výraz, kde A je pole \texttt{a}.
\begin{ceqn}
    \begin{align*}
        A[i] \rightarrow \langle \exists j: j \geq i :: A[i] \geq A[j] \rangle
    \end{align*}
\end{ceqn}

V krokoch 1.29, 1.37 a 1.69 môžeme vidieť, že existuje také $A[i]$, ktoré je väčšie ako 
$A[j]$. A keďže tento výraz platí pre niekoľko krokov programu vlastnosť \textbf{leads-to je 
splnená}.