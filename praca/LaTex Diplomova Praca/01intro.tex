\chapter{Úvod}\label{chap:intro}
UNITY je teoretický jazyk vytvorený za účelom riešenia paralelných problémov. V dnešnej 
dobe existuje iba málo takých paralelných programovacích jazykov, ktoré sa snažia 
dané problémy vyriešiť. Avšak neobsahujú všetky potrebné prvky na ich úplne vyriešenie. 
Preto je tu návrh za vytvorenie verifikačného nástroju podporujúci zápis a 
verifikáciu programov zapísaných v jazyku UNITY. 

V nasledujúcich kapitolách tejto diplomovej práce sa dozviete všetky potrebné 
informácie o samotnom programe UNITY, návrhu a implementácie interpretera, ktorý je 
jadrom verifikačného nástroju, skúmanie použitých nástrojov a knižníc na dosiahnutie 
požadovaných výsledkov.