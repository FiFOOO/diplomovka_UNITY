\chapter{Interpreter UNITY}\label{chap:unity}

Táto kapitola obsahuje popis postupu vytvárania interpretera pre verifikačný nástroj UNITY, prieskum 
dostupných knižníc na verifikáciu CTL logiky a ich základné funkcie a vlastnosti.

\section{Základné informácie}
V predchádzajúcej kapitole sme sa dozvedeli základné pojmy interpretera, knižníc, ktoré sú k dispozícií a samotného 
programu UNITY. Teraz sme všetky tieto znalosti uviedli do teoretického plánu pre verefikačný 
nástroj pre UNITY. Čo sme si predstavovali pod pojmom verifikačný nástroj? Išlo o nástroj, ktorý mal kontrolovať, či daný
program spĺňa všetky spomenuté vlastnosti programu (FP, Unless, Ensures, Leads-to). K dosiahnutiu potrebných výsledkov bolo za potrebu 
vytvoriť pre verifikačný nástroj interpreter, ktorý daný program načítal a spracoval. Vrátené dáta boli 
vyhodnotené na základe použitej knižnice zo zbierky nástrojov, ktoré poskytuje LTSmin. 

\section{Úprava jazyku UNITY}
Pre ciele tejto práce bolo potrebné zmeniť niektoré syntaktické prvky a vlastnosti jazyku UNITY:

\begin{itemize}
    \item Náhodné hodnoty neinicializovanej premennej
    \item Syntax always seckcie
    \item Syntax program seckie
\end{itemize}

\subsection{Náhodné hodnoty neinicializovanej premennej}
Ako už vieme ak je deklarovaná premenná ale nie je inicializovaná podľa vlastností UNITY by mala táto 
premenná nadobudnúť náhodnú hodnotu, avšak pri riešení niektorých úloh by táto vlastnosť mohla dané 
výsledky programu znehodnotiť. Problém nastával v rozsahu náhodných čísiel. Ak bola daná premenná 
deklarovaná ako integer nevedeli sme presne určiť najsprávnejšie riešenie pre tento rozsah čísiel. Preto 
sme túto vlastnosť upravili nasledovne. V implementácií bolo potrebné každú deklarovanú premennú inicializovať. 
Týmto sme sa úplne vyhli náhodne generovaným hodnotám pre deklarované premenne.

\subsection{Syntax always sekcie}
Sekcia always slúži na vytvorenie tzv. transparentných premenných, ktoré sú definované názvom premennej na ľavej strane 
a na pravej strane sú jednotlivé výrazy. Toto priradenie bolo definované znakom rovnosti \texttt{=}. V našom prípade, aby parser
interpretera nemal problém rozlišovať medzi operátormi \texttt{=>, <=, !=, ==} a znakom rovnosti sme znak rovnosti 
zmenili na \texttt{:=} tak isto ako tomu je v assign sekcii.

\subsection{Syntax program sekcie}
V ukážke programu máme sekciu program, tá obsahuje program-name, názov programu, ktorý sa môže vynechať, lenže 
aby bol program kompletný zo strany syntaxe bola táto zložka povinná.

\section{Postup pri vytváraní interpretera}

\subsection{Proces fungovania}
Interpreter je jadro celého verifikačného nástroju pre programovací jazyk UNITY. 
Obsahuje tieto časti:

\subsubsection{Spracovanie vstupného textu - telo programu}
Interpreter vstupný text prečíta, rozdelí na tokeny, ktoré parser spracuje, skontroluje syntax a vyhodnotí 
či daný program neobsahuje chybu. Výsledkom parseru bude vygenerovanie abstraktného syntaktického stromu - AST.

\subsubsection{AST}
Vygenerovaný abstraktný syntaktický strom obsahuje celé telo programu. Koreň stromu je program a 
listy sú sekcie programu, tie obsahujú všetky priradenia danej sekcie. Algoritmus, ktorý 
tento strom prechádza vygeneruje cieľový kód pre zvolenú knižnicu.

\subsubsection{Cieľový kód}
Tento kód musí spĺňať všetky vlastnosti zvolenej knižnice. Každá z knižníc LTSmin využíva odlišný 
paralelný programovací jazyk alebo zápis programu do špecifického modelu. Preto je potrebné cieľový kód 
upraviť podľa syntaktických pravidiel zvolenej knižnice. Výsledkom kódu je program sformalizovaný podľa 
zvolenej knižnice.

\subsubsection{Vykonanie kódu}
Knižnice obsahujú nástroje, ktoré vygenerovaný kód vykonajú. Výsledkami sú vlastnosti programu.

\subsection{Skúmanie dostupných knižníc LTSmin}
Pri výbere bolo viacero možností knižníc, ktoré zvoliť. Hlavnou podmienkou ale bolo aby knižnica spĺňala 
temporálnu logiku CTL. 

\subsubsection{DiViNe}
Jedným z príkladov bol model checker DiViNe, ktorý je efektívny aj na veľmi náročné 
logické formule. DiViNe nie je náročný na výpočtovú silu a preto je vhodný na používanie na jednom 
počítači alebo sieti výpočtových staníc. Model môže byť vytvorený pomocou DVE modelovacieho jazyka. 
Avšak DiViNe je hlavne zameraný na temporálnu logiku LTL aj keď z istej časti dokáže pracovať s CTL 
nie je pre nás vhodný model checker, pretože aj zapisovanie do jazyku DVE nie je jednoduché a dostupná 
dokumentácia neobsahuje dostatok informácií.

\subsubsection{opaal}
Tento model checker zaručuje ciele, pre ktoré sa oplatí používať ako je napr.
Rapid prototyping a Easy to learn. Je písaný v jazyku Python, čo by nám niekoľko vecí 
uľahčilo ale jeho hlavnými obmedzeniami sú, že dokáže pracovať iba s časovými automatmi 
a je implementovateľný iba na Ubuntu 12.04 a Ubuntu 14.04.

\subsubsection{SpinS}
SpinS je model checker, ktorý bol pôvodne vytvorený z knižnice SpinJa, písaná v jazyku JAVA. Obe tieto 
knižnice sú pôvodne vytvorené z model checker-u Spin, ktorým je základom paralelný programovací jazyk
Promela. Tento jazyk je jednoduchý a vysoko efektívny na logiku LTL. Hlavnou podmienkou práce bolo
aby daná knižnica vedela pracovať na úrovni logiky CTL. To sa vďaka tomuto jazyku dá jednoducho docieliť.
V prípade, že každé jedno priradenie z jazyku UNITY vložíme do nekonečného cyklu, bežiaceho vlákna zaručíme 
všetky potrebné vlastnosti pre logiku CTL a zároveň jazyku UNITY. Ak uvažujeme nad tým, že UNITY program 
sa skladná z množiny priradení a každé toto priradenie vložíme do samostatného vlákna, ktoré
sa bude vykonávať nepretržite v istom okamihu sa zaručene dostaneme do FP a dosiahneme 
požadovaných výsledkov. Ibaže pri overovaní vlastnosti nám nastali komplikácie a z tohoto dôvodu sme danú 
knižnicu nevyužili na overovanie vlastností, ale iba testovanie a debugovanie programov písaných v Promela.

\section{Promela - Spin}
\subsection{Promela}
Je to verifikačný programovací jazyk zameraný na tvorbu modelov pre Spin model checker. Jeho základnými
vlastnosťami sú:

\begin{itemize}
    \item asynchrónne vykonávanie programu
    \item zdieľané premenné
    \item komunikačné kanále (channels)
\end{itemize}

Jeho úlohou je overiť logiku paralelných systémov. Na to slúži Spin, ktorý daný program vykoná. 
Nasledujúca tabuľka (\ref{obr:tabdatatypes}) zobrazuje dátové typy, ktoré Promela podporuje.

\begin{center}
    \begin{tabular}{|c|c|c|c|}
    \hline
    \textbf{Name} & \textbf{Size (bits)} & \textbf{Usage} & \textbf{Range} \\ \hline
    bit           & 1                    & unsigned       & 0..1           \\ \hline
    bool          & 1                    & unsigned       & 0..1           \\ \hline
    byte          & 8                    & unsigned       & 0..255         \\ \hline
    mtype         & 8                    & unsigned       & 0..255         \\ \hline
    short         & 16                   & signed         & -215..215 - 1  \\ \hline
    int           & 32                   & signed         & -215..215 - 1  \\ \hline
    \end{tabular}
    \begin{table}[!h]
		\caption[Dátové typy]{Dátové typy}
		\label{obr:tabdatatypes}
	\end{table}
\end{center}

\subsubsection{Popis jazyku Promela}
Operátory a priradenia sú veľmi podobné ako v jazyku C alebo JAVA. Premenné musia byť definované
typovo a jednotlivé priradenia oddelené bodkočiarkou.

\begin{lstlisting}
    int x = 10;
    int a, b = 1, 2;
\end{lstlisting}

Taktiež môžu byť definované globálne premenné, alebo funkcie podobne ako v jazyku C za pomoci 
\texttt{define}.

\begin{lstlisting}
    #define N 10
    #define sum(a, b) ((a) + (b))
\end{lstlisting}

Ak chceme takúto funkciu definovať viacriadkovo použijeme \texttt{inline}.

\begin{lstlisting}
    inline sum(a,b) {
        a + b
    }
\end{lstlisting}

V Promela je možné vytvoriť proces \texttt{proctype}, ktorý je buď definovaný ako funkcia a 
následne spustená v \texttt{init} alebo priamo funkciu napísať a priamo ju pustiť za pomoci 
\texttt{active}.

\begin{lstlisting}
    byte n=0;
    active proctype P() {
        n = 1;
        printf("Process P, n = %d\n", n);
    }
    proctype Q() {
        n = 2;
        printf("Process Q, n = %d\n", n);
    }
    init {
        run Q();
    }
\end{lstlisting}

Je tiež možné vykonávať atomické operácie za pomoci \texttt{atomic} do ktorého stačí obaliť operácie, 
ktoré chceme vykonávať atomicky.

\begin{lstlisting}
    active proctype P() {
        atomic {
            n = 1;
            printf("Process P, n = %d\n", n);
        }
    }
\end{lstlisting}

Cykly sa vykonávajú za pomoci \texttt{do}, ktoré obsahujú podmienky a v prípade, že sú \texttt{true}
vykonajú sa priradenia. Každá nová podmienka a priradenie musí byť oddelené \texttt{::}. Takýto cyklus
bude pokračovať nepretržite pokiaľ nebude zastavený pomocou \texttt{break}. 

\begin{lstlisting}
    do
    :: condition -> statement
    ...
    :: a >= b -> max = a
    od
\end{lstlisting}

Podmienené operácie uvedené v \texttt{if} sa vykonávajú nedeterministicky. 

\begin{lstlisting}
    if
    :: condition -> statement
    ...
    :: a >= b -> max = a; branch = 1
    :: a <= b -> max = b; branch = 2
    fi;
\end{lstlisting}

Objekty sa vytvárajú nasledovne.

\begin{lstlisting}
    typedef MyStruct
    {
        short Field1;
        byte  Field2;
    }

    init {
        MyStruct x;
        x.Field1 = 1;
    }
\end{lstlisting}

\subsection{Spin}
Je verifikačný nástroj, ktorý podporuje dizajn a verifikáciu asynchrónnych procesov. Verifikačné modely
sú zamerané na dokazovanie správnosti iterácií procesov \cite{br9}. V stručnosti nám Spin slúžil
ako verifikátor pre jazyk Promela. Pomocou neho sme boli schopný testovať a debugovať jednotlivé programy.
Taktiež nám vedel poskytnúť základné informácie o danom testovanom algoritme ako je napr. počet stavov, procesov
a premenných použitých v programe. V našom prípade nám to pomohlo či je daný program napísaný správne a efektívne.
Príklad programu:

\begin{lstlisting}
    mtype = {red, yellow, green};
    mtype light=green;
    init {
        do
        :: if
        :: light==red -> light=green
        :: light==yellow -> light=red
        :: light==green -> light=yellow
        fi;
        od
    }
\end{lstlisting}

Daný program bolo možné spustiť príkazom \texttt{./spin tlight.pml}. Kde \texttt{tlight.pml} je súbor s programu
s koncovkou \texttt{pml} známu pre Promela. Avšak v tomto prípade sa program zacyklí a to z dôvodu, že program
bol spustený v tzv. simulation mode. Pre plné využitie model checker-a je potrebné program spustiť v tzv. exhaustive 
verification mode, kedy využijeme všetky schopnosti model checker-a. K takémuto spusteniu sme mali dve možnosti:

\begin{lstlisting}
    spin -a tlight.pml
    gcc -o pan pan.c
    pan
\end{lstlisting}

alebo 

\begin{lstlisting}
    spin -search tlight.pml
\end{lstlisting}

V prvom prípade nám Spin vytvoril súbor \texttt{pan.c} v jazyku C, ten sme museli následne skompilovať príkazom 
\texttt{gcc} a vráti nám spustiteľný súbor \texttt{pan}. V druhom prípade nám príkaz vráti priamo to isté ako 
súbor \texttt{pam} a tým je výsledok verifikácie programu. Ak pri verifikácií nastal akýkoľvek error vytvorí
sa súbor \texttt{trail}, ktorý popisuje všetky informácie o prípadnom deadlocku alebo vzniknutej chybe. Do súboru 
\texttt{trail} je možné nahliadnuť príkazom \texttt{spin -t -p -l tlight.pml}. Ak by program obsahoval 
príkaz \texttt{printf} vydeli by sme niekoľko veľa výstupov.

\begin{lstlisting}
    ...
    printf("The light is now %e\n",light);
    ...
\end{lstlisting}

Aby sme tomuto zabránili a otestovali výstupy bolo potrebné obmedziť počet krokov vykonaných 
v programe a to príkazom \texttt{spin -u40 tlight.pml}, ktorý vykoná iba 40 krokov.

\section{Výsledné použité knižnice}
Po celkovom skúmaní knižníc sme dospeli k záveru, že ani jeden zo spomínaných model checher-ov nevyhovuje 
nášmu cieľu. Preto týmto skúmanie neskončilo a výsledné použité knižnice boli S2N, NuSMV a programovací 
jazyk Promela s pomocou Spin.

\subsection{NuSMV}
NuSMV je výsledkom reengineeringu a reimplementácie symbolického model checker-u SMV. NuSMV 
bol aktualizovaný v troch častiach oproti SMV:

\begin{itemize}
    \item NuSMV má viacej funkcií, napr. CTL a LTL verfikáciu, ktoré zvyšujú schopnosť 
    používateľa spolupracovať so systémom.
    \item Systémová architektúra NuSMV je vysoko modulárna a otvorená. Ďalšou vlastnosťou je, 
    že v NuSMV môže užívateľ riadiť a prípadne meniť poradie vykonávania niektorých systémových modulov.
    \item Kvalita implementácie sa výrazne zvyšuje. NuSMV je veľmi robustný a dobre zdokumentovaný systém, 
    ktorého kód je (relatívne) ľahko modifikovateľný.
\end{itemize}

NuSMV dokáže spracovať súbory napísané v jazyku SMV a umožňuje konštrukciu modelu s rôznymi metódami, 
analýzou dosiahnuteľnosti, kontrolou modelu CTL, výpočtom kvantitatívnych charakteristík 
modelu a generovaním protikladov \cite{br8}. Základné príkazy pre NuSMV:

\begin{itemize}
    \item \texttt{NuSMV -int [filename].smv} - načíta daný súbor do interaktívneho shell-u
    \item \texttt{go} - prečíta a inicializuje NuSMV pre verifikáciu a simuláciu
    \item \texttt{pick\_state [-v] [-r | -i]} - vyberie stav z množiny stavov
    \begin{itemize}
        \item[\textbullet] \texttt{-v} vypíše vybraný stav
        \item[\textbullet] \texttt{-r} vyberie náhodný stav
        \item[\textbullet] \texttt{-i} stav si vyberieme sami
    \end{itemize}
    \item \texttt{simulate [-r | -i] -k steps} - od aktuálneho stavu vygeneruje počet stavov zadaných 
    prepínačom \texttt{-k}
    \begin{itemize}
        \item[\textbullet] \texttt{-r} každý nový vygenerovaný stav zvolí náhodne
        \item[\textbullet] \texttt{-i} každý nový vygenerovaný stav si vyberieme sami
    \end{itemize}
    \item \texttt{show\_traces -v} - vypíše všetky vykonané kroky
    \item \texttt{reset} - resetuje celý systém
    \item \texttt{quit} - ukončí shell
\end{itemize}

Keďže NuSMV podporuje verifikáciu CTL zvolili sme si práve tento model checker na overovanie vlastností 
jazyku UNITY. Bude hlavnou súčasťou verifikátora v našej aplikácií. Avšak NuSMV vie pracovať iba s jazykom SMV, 
ktorý je na zápis UNITY programu príliš zložítý. Preto použijeme nástroj S2N na transformáciu z jazyka Promela
na jazyk SMV.

\subsection{S2N}
Po skúmaní verifikačných nástrojov v zbierke LTSmin sme nenašli žiaden nám vyhovujúci nástroj. Väčšina týchto
nástrojov nespĺňala požiadavku aby takýto verifikačný model checker dokázal overovať CTL logiku.
Po hlbšom pátraní sme ale narazili na jeden nástroj, ktorý dokáže z modelu spísanom v jazyku Promela resp. Spin 
vytvoriť CTL model. Tento nástroj sa volá S2N - Spin to NuSMV. 

Spin a NuSMV sú dva najviac rozšírené model checker-y. Spin používa jazyk Promela, ktorý je určený 
na modelovanie synchronizácie a koordinácie procesov. Promela má syntax podobnú jazyku C, čo uľahčuje vytváranie
čítanie modelov. NuSMV poskytuje jazyk na opis stavových prechodných systémov. Spin a NuSMV sú veľmi pôsobivé 
a úspešné vo svojich oblastiach. Zosúladenie ich výhod by bolo užitočné pre modelovanie. Preto vznikol
nástroj S2N, ktorý dokáže preložiť modely v jazyku Promela do jazyka NuSMV. Toto spojenie prináša veľa 
výhod. Na jednej strane je modelovanie zložitého systému v NuSMV tvrdou prácou, ktorá je náchylná na chyby. 
Teraz máme pre model NuSMV jazyk vyššej úrovne. Na druhej strane S2N funguje tak, že rozširuje systém 
Spin o schopnosť kontrolovať CTL a ďalšie vlastnosti NuSMV \cite{br7}.

S2N podporuje väčšinu základných funkcií jazyku Promela. Ako sú napr. polia, všetky dátové typy, do a if, 
proctype a aj komunikáciu pomocou kanálov. Avšak nepodporuje inline, typedef a printf, ktoré sme pre naše účely
aj tak nepotrebovali. Všetky dostupné vlastnosti S2N nám vyhovovali na transformovanie Promela na NuSMV.