\documentclass[12pt, a4paper, oneside]{book}
\usepackage[hidelinks]{hyperref}
\usepackage[slovak]{babel}
\usepackage{epsfig}
\usepackage{epstopdf}
\usepackage[chapter]{algorithm}
\usepackage{algorithmic}
\usepackage{listings}
\usepackage{amsmath}
\usepackage{nccmath}
\usepackage{amssymb}
\usepackage{graphicx}
\usepackage{multirow}
\usepackage{color}
\usepackage{url}
\usepackage[utf8]{inputenc}
\usepackage[T1]{fontenc}
\usepackage{setspace}
\usepackage{tabularx}
\usepackage{textcomp}
\usepackage{caption}
\usepackage{natbib}
\usepackage{enumitem}
\usepackage{graphicx}
% \usepackage{geometry}


%  \geometry{
%  a4paper,
%  total={170mm,257mm},
%  top=25mm,
%  bottom=25mm,
%  left=35mm,
%  right=20mm
% }


\def\infinity{\rotatebox{90}{8}}

\lstset{frame=tb,
  language=C,
  aboveskip=3mm,
  belowskip=3mm,
  showstringspaces=false,
  columns=flexible,
  basicstyle={\small\ttfamily},
  numbers=none,
  breaklines=true,
  breakatwhitespace=true,
  tabsize=3,
  inputencoding=utf8,
  texcl=true
}

\setstretch{1.5}
%\renewcommand\baselinestretch{1.5} % riadkovanie jeden a pol

% pekne pokope definujeme potrebne udaje
\newcommand\mftitle{Verifikačný nástroj pre formalizmus UNITY}
\newcommand\mfthesistype{Diplomová práca}
\newcommand\mfauthor{Bc. Filip Špaldoň}
\newcommand\mfadvisor{doc. RNDr. Damas Gruska, PhD.}
\newcommand\mfplacedate{Bratislava, 2019}
\newcommand\mfuniversity{UNIVERZITA KOMENSKÉHO V BRATISLAVE}
\newcommand\mffaculty{FAKULTA MATEMATIKY, FYZIKY A INFORMATIKY}
\newcommand{\sub}[1]{$_{\text{#1}}$}
\newcommand{\reference}[1]{č.~\ref{#1}}
\newcommand{\imageHeight}{150px}

\ifx\pdfoutput\undefined\relax\else\pdfinfo{ /Title (\mftitle) /Author (\mfauthor) /Creator (PDFLaTeX) } \fi

\begin{document}

\frontmatter

\thispagestyle{empty}

\noindent
\begin{minipage}{\textwidth}
\begin{center}
\textbf{\mfuniversity \\
\mffaculty}
\end{center}
\end{minipage}

\vfill
\begin{figure}[!hbt]
	\begin{center}
		\includegraphics{images/logo_fmph}
		\label{img:logo}
	\end{center}
\end{figure}
\begin{center}
	\begin{minipage}{0.8\textwidth}
		\centerline{\textbf{\Large\MakeUppercase{\mftitle}}}
		\smallskip
		\centerline{\mfthesistype}
	\end{minipage}
\end{center}
\vfill
2019 \hfill
\mfauthor
\eject 
% koniec obalu

\thispagestyle{empty}

\noindent
\begin{minipage}{\textwidth}
\begin{center}
\textbf{\mfuniversity \\
\mffaculty}
\end{center}
\end{minipage}

\vfill
\begin{figure}[!hbt]
\begin{center}
\includegraphics{images/logo_fmph_dark}
\label{img:logo_dark}
\end{center}
\end{figure}
\begin{center}
\begin{minipage}{0.8\textwidth}
\centerline{\textbf{\Large\MakeUppercase{\mftitle}}}
\smallskip
\centerline{\mfthesistype}
\end{minipage}
\end{center}
\vfill
\begin{tabular}{l l}
%Registration number: & 40a99bd8-3cb6-4534-9330-c7fd9b5e5ca4 \\
Študijný program: & Aplikovaná informatika\\
Študijný odbor: & 2511 Aplikovaná informatika\\
Školiace pracovisko: & Katedra aplikovanej informatiky\\
Školiteľ: & \mfadvisor
\end{tabular}
\vfill
\noindent
\mfplacedate \hfill
\mfauthor
\eject 
% koniec titulneho listu

%\thispagestyle{empty}
%\includegraphics[width=\textwidth]{images/zadanie}
%\vfill
%\eject
% koniec zadania

\thispagestyle{empty}


%\begin{figure}[H]
%\begin{center}
%\makebox[\textwidth]{\includegraphics[width=\paperwidth]{images/zadaniedp}}
%\label{img:zadanie}
%\end{center}
%\end{figure}


{~}\vspace{12cm}

\noindent
\begin{minipage}{0.25\textwidth}~\end{minipage}
\begin{minipage}{0.75\textwidth}
Čestne prehlasujem, že túto diplomovú prácu som vypracoval samostatne len s použitím uvedenej literatúry a za pomoci konzultácií u môjho školiteľa.
\newline \newline
\end{minipage}
\vfill
~ \hfill {\hbox to 6cm{\dotfill}} \\
\mfplacedate \hfill \mfauthor
\vfill\eject 
% koniec prehlasenia

\chapter*{Poďakovanie}\label{chap:thank_you}
Ďakujem školiteľovi doc. RNDr. Damasovi Gruskovi, PhD. za rady a pomoc počas riešenia mojej diplomovej práce. 
Ďakujem aj svojím blízkym za morálnu podporu.

%
%\vfill\eject 
%% koniec podakovania
%
\chapter*{Abstrakt}\label{chap:abstract_sk}
\textbf{U}nbounded \textbf{N}ondeterministic 
\textbf{I}terative \textbf{T}ransformations of the program state (UNITY) je programovací jazyk popísaný v knihe 
Parallel Program Design - A Foundation a jeho autormi sú K. Mali Chandy a Jayadev Misra z University of Texas. 
Jeho hlavnými znakmi sú nedeterminizmus, paralelné operácie, absencia control-flow, synchrónnosť
a asynchrónnosť, stavy a priradenia. 

Všetky tieto znaky bolo potrebné aplikovať do vytvoreného interpretera, ktorý sa staral o činnosť
verifikátoru. Práca popisuje podrobnú implementáciu lexikálneho a syntaktického analyzátoru interpretera.
Tiež obsahuje popis skúmaných knižníc a nástrojov LTSmin, výsledné použité knižnice S2N a NuSMV.
Výsledný verifikátor dokáže pracovať s logikou výpočtového stromu (CTL). Všetky tieto ciele sme mohli 
dosiahnuť vďaka webovej aplikácií postavenej na platformách GO, React, ktoré boli použité 
na implementáciu verifikátora.

\paragraph*{Kľúčové slová}: UNITY, CTL, LTSmin, S2N, NuSMV, intepreter, verifikátor


\chapter*{Abstrakt EN}\label{chap:abstract_en}
\textbf{U}nbounded \textbf{N}ondeterministic 
\textbf{I}terative \textbf{T}ransformations of the program state
(UNITY) is a programming language described in the book Parallel Program Design -
A Foundation and the authors are K. Mali Chandy and Jayadev Misra from the University of 
Texas. Its main features are nondeterminism, paralell operations, absence of control flow, 
synchronity and asynchronity, states and assignments.

All these features were needed to be applied to the created interpreter that was 
managing the action of the verificator. The thesis describes a detailed implementation 
of the lexical and syntactic analyser of the interpreter. It also contains a description 
of the studied libraries and LTSmin tools, the resultant used libraries S2N and NuSMV. 
The resultant verificator is able to work with the logics of computational tree (CTL). 
All these aims were reached thanks to a web application based on the platform GO, React, 
which were used for the implementation of the verificator.

\paragraph*{Keywords}: UNITY, CTL, LTSmin, S2N, NuSMV, intepreter, verificator
%
%\vfill\eject 
%
%
%\vfill\eject 
%% koniec abstraktov

\tableofcontents
\listoffigures
\listoftables

\mainmatter

% treba este prejst dokument ci je kod spravne formatovany
\input 01intro.tex
% \input 02motivation.tex
\input 03issues_overview.tex
\input 04previous_solutions.tex
% \input 05proposal.tex
\input 06implementation.tex
\input 07results.tex
\input 08conclusion.tex

\backmatter

\nocite{*}
\bibliographystyle{alpha}
%\bibliography{references}

\begin{thebibliography}{5}
 
\bibitem{br1}  H. Conrad Cunningham, Gruia-Catalin Roman. A UNITY - Style Programming Logic for a Shared Dataspace Language, 1989 

\bibitem{br2}  Lawrence C. Paulson. Mechanizing UNITY in Isabelle, 2000

\bibitem{br3}  PC Mag Staff. Encyclopedia: Definition of Compiler, 28 February 2017

\bibitem{br4}  Baier C., Katoen J. Principles of Model Checking, 2008

\bibitem{br5}  Gijs Kant, Alfons Laarman, Jeroen Meijer, Jaco van de Pol, Stefan Blom1, Tom van Dijk. LTSmin: High-Performance Language-Independent Model Checking, 2015

\bibitem{br6}  Michal Šuster. Intepreter UNITY, 2006

\bibitem{br7} Yong Jiang, Zongyan Qiu. S2N: Model Transformation from SPIN to NuSMV, 2012

\bibitem{br8} A. Cimatti, E. Clarke, F. Giunchiglia, M. Roveri. NuSMV: A New Symbolic Model Verifier, 1999

\bibitem{br9} Gerard J. Holzmann. The Model Checker SPIN, May 1997

\bibitem{br10} Vladimír Kvasnička. Jiří Pospíchal, Matematická logika, 2006

\bibitem{br11} Martin Baláž. Dynamické Kripkeho štruktúry pre dobre založenú sémantiku, 2002

\end{thebibliography}

\end{document}